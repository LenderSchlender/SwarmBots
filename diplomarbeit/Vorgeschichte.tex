% Vorgeschichte und Prototyp aka. Projektwoche 2023/24
\chapter{Vorgeschichte}
\label{sec:vorgeschichte}
Im Zuge der Projektwoche im Schuljahr 2023/24 haben wir bereits begonnen,
einen ersten Prototypen unseres SwarmBots-Systems zu bauen.
%
Dieser Prototyp bestand aus nur zwei Robotern,
welche auf Basis eines fertigen Fahrgestells zu sehr wackligen Gefährten wurden.
%
Sehr viel Autonomie hatten die früheren Roboter auch noch nicht,
der LiDAR war noch nicht funktionstüchtig,
stattdessen wurden die Roboter mittels Videospiel-Controller ferngesteuert.
%
Diese Fernsteuerung wurde aber auch schon damals über eine Websocket-Verbindung implementiert.
%
Außerdem wurde während der Projektwoche der Code für die ESP32-CAMs fast komplett fertiggestellt,
diese verwenden jetzt immer noch größtenteils das gleiche Programm.
%
Trotz der Schwächen des damaligen Systems wurde unserem Projekt
der erste Preis in der Kategorie ``Experten'' verliehen.

% TODO Bilder zur Projektwoche

\section{Projektmanagment}
\label{subsec:projektmanagment}
%
\subsection{Projektplanung}
%
Zu Beginn des Projekts wurde ein Projektziel definiert:
Es sollten mehrere Roboter gebaut werden, die als Schwarm zusammenarbeiten können. Dafür mussten die Roboter modifiziert und mit einem zentralen Server verbunden werden.
Die Projektplanung bestand aus folgenden Schritten:
\begin{itemize}
    \item Aufgabenverteilung im Team
    \item Erstellung eines Zeitplans mit allen wichtigen Meilensteinen
    \item Planung des Materialbedarfs (Kits, Sensoren, Bauteile)
    \item Organisation der Hardware- und Softwareentwicklung
\end{itemize}
Die Arbeitspakete wurden in einem Projektstrukturplan (PSP) festgehalten.
Jedes Teammitglied war für bestimmte Bereiche verantwortlich, z.B. Hardwareaufbau, Softwareprogrammierung oder Dokumentation.
%
\subsection{Controlling}
Um den Projektverlauf zu überwachen, wurde ein Controlling-System eingeführt. Dieses bestand aus drei Bereichen:
\subsubsection{Zeit-Controlling}
Es wurde ein Zeitplan mit Abgabeterminen und Zwischenschritten erstellt.
Die wichtigsten Termine waren:
\begin{itemize}
    \item Einkauf und Aufbau der Roboter
    \item Fertigstellung der Hardware-Modifikationen
    \item Abschluss der Softwareentwicklung
    \item Durchführung der Tests
    \item Abgabe der Dokumentation
\end{itemize}
Der Zeitplan wurde regelmäßig kontrolliert und angepasst, wenn es zu Verzögerungen kam.
\subsubsection{Kosten-Controlling}
Für das Projekt wurde eine Kostenaufstellung erstellt. Alle Ausgaben für Bauteile, Werkzeuge und Software wurden darin dokumentiert.
Die wichtigsten Kostenpunkte waren:
\begin{itemize}
    \item Elegoo Tumbller Kits
    \item ESP32-Boards
    \item LiDAR-Sensor
    \item Zubehör (Kabel, Steckverbinder, etc.)
\end{itemize}
Am Ende wurde geprüft, ob das Projektbudget eingehalten wurde.
\subsubsection{Qualitäts-Controlling}
Die Qualität der Hardware und Software wurde während des gesamten Projekts durch Tests und Prüfungen sichergestellt.
Es wurde besonders darauf geachtet, dass:
\begin{itemize}
    \item alle Sensoren zuverlässig arbeiten
    \item die Kommunikation zwischen Roboter und Server stabil ist
    \item der Aufbau sauber und sicher ausgeführt wurde
\end{itemize}
%
\subsection{Problemlösung und Risikomanagement}
%
Während des Projekts traten verschiedene Probleme auf, die gelöst werden mussten:
Problem	Lösung
Bluetooth-Reichweite zu gering	Umstieg auf WLAN mit ESP32-Boards
Platzmangel für LiDAR-Sensor	Mechanische Anpassung des Guide-Roboters
Signalstörungen bei den Motoren	Überprüfung und Neuanordnung der Kabel
Instabile Verbindung bei hoher Serverlast	Optimierung der Websocket-Kommunikation
Fehlermeldungen in der Software	Implementierung von Debug-Tools und Log-Funktion
Verzögerungen bei der Hardwarelieferung	Anpassung des Zeitplans und Priorisierung der Arbeit
Alle Probleme wurden im Team besprochen und gelöst. Bei größeren Schwierigkeiten wurde ein Risiko-Logbuch geführt.
\subsection{Software-Management (SW)}
%
Die Softwareentwicklung war ein wichtiger Bestandteil des Projekts. Sie bestand aus folgenden Bereichen:
\subsubsection{Entwicklungsschritte}

\begin{enumerate}
    \item \texttt{Server-Software:} \\
    Die Server-Software wurde mit Node.js und Websocket-Technologie umgesetzt. Sie sammelt die Sensordaten der Roboter und verteilt sie an alle Teilnehmer.
    \item \texttt{Roboter-Software:} \\
    Die Roboter wurden mit einer Arduino-IDE programmiert. Die ESP32-Boards verwenden eine WLAN-Verbindung und kommunizieren über ein einfaches TCP/IP-Protokoll.
    \item \texttt{Frontend (Benutzeroberfläche):} \\
    Zusätzlich wurde eine Webseite entwickelt, auf der die Positionen und Daten der Roboter in Echtzeit angezeigt werden.
  \end{enumerate}
\subsubsection{Versionskontrolle}
Für die Software wurde ein Git-Repository verwendet. Damit konnten Änderungen dokumentiert, rückgängig gemacht und im Team bearbeitet werden.
\subsubsection{Tests and Debugging}
Die Software wurde regelmäßig getestet:
\begin{itemize}
    \item Unit-Tests für einzelne Funktionen
    \item Integrationstests für die Kommunikation
    \item Live-Tests mit allen Robotern
\end{itemize}
\subsection{Dokumentation und Abschluss}
Zum Abschluss des Projekts wurden alle Ergebnisse in einer Diplomarbeit dokumentiert.
Dazu gehören:
\begin{itemize}
    \item Die Beschreibung der Hardware
    \item Die Modifikationen
    \item Die Software-Struktur
    \item Der Projektverlauf
\end{itemize}

