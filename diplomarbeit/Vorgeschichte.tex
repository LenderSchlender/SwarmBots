% Vorgeschichte und Prototyp aka. Projektwoche 2023/24
\chapter{Vorgeschichte}
\label{sec:vorgeschichte}
Im Zuge der Projektwoche im Schuljahr 2023/24 haben wir bereits begonnen,
einen ersten Prototypen unseres SwarmBots-Systems zu bauen.
%
Dieser Prototyp bestand aus nur zwei Robotern,
welche auf Basis eines fertigen Fahrgestells zu sehr wackligen Gefährten wurden.
%
Sehr viel Autonomie hatten die früheren Roboter auch noch nicht,
der LiDAR war noch nicht funktionstüchtig,
stattdessen wurden die Roboter mittels Videospiel-Controller ferngesteuert.
%
Diese Fernsteuerung wurde aber auch schon damals über eine Websocket-Verbindung implementiert.
%
Außerdem wurde während der Projektwoche der Code für die ESP32-CAMs fast komplett fertiggestellt,
diese verwenden jetzt immer noch größtenteils das gleiche Programm.
%
Trotz der Schwächen des damaligen Systems wurde unserem Projekt
der erste Preis in der Kategorie ``Experten'' verliehen.

% TODO Bilder zur Projektwoche

\section{Projektmanagment}
\label{subsec:projektmanagment}
%
\subsection{Projektplanung}
%
Zu Beginn des Projekts wurde ein Projektziel definiert:
Es sollten mehrere Roboter gebaut werden, die als Schwarm zusammenarbeiten können. Dafür mussten die Roboter modifiziert und mit einem zentralen Server verbunden werden.
Die Projektplanung bestand aus folgenden Schritten:
•	Aufgabenverteilung im Team
•	Erstellung eines Zeitplans mit allen wichtigen Meilensteinen
•	Planung des Materialbedarfs (Kits, Sensoren, Bauteile)
•	Organisation der Hardware- und Softwareentwicklung
Die Arbeitspakete wurden in einem Projektstrukturplan (PSP) festgehalten.
Jedes Teammitglied war für bestimmte Bereiche verantwortlich, z.B. Hardwareaufbau, Softwareprogrammierung oder Dokumentation.
%
\subsection{Controlling}
%
\subsection{Problemlösung und Risikomanagement}
%
\subsection{Software-Management (SW)}
%
\subsection{Dokumentation und Abschluss}