% Vorgeschichte und Prototyp aka. Projektwoche 2023/24
\chapter{Vorgeschichte}
\label{sec:vorgeschichte}
Im Zuge der Projektwoche im Schuljahr 2023/24 haben wir bereits begonnen,
einen ersten Prototypen unseres SwarmBots-Systems zu bauen.
%
Dieser Prototyp bestand aus nur zwei Robotern,
welche auf Basis eines fertigen Fahrgestells zu sehr wackligen Gefährten wurden.
%
Sehr viel Autonomie hatten die früheren Roboter auch noch nicht,
der LiDAR war noch nicht funktionstüchtig,
stattdessen wurden die Roboter mittels Videospiel-Controller ferngesteuert.
%
Diese Fernsteuerung wurde aber auch schon damals über eine Websocket-Verbindung implementiert.
%
Außerdem wurde während der Projektwoche der Code für die ESP32-CAMs fast komplett fertiggestellt,
diese verwenden jetzt immer noch größtenteils das gleiche Programm.
%
Trotz der Schwächen des damaligen Systems wurde unserem Projekt
der erste Preis in der Kategorie ``Experten'' verliehen.

% TODO Bilder zur Projektwoche

\section{Projektmanagment}
\label{subsec:projektmanagment}
%
\subsection{Projektplanung}
%
Zu Beginn des Projekts wurde ein Projektziel definiert:
Es sollten mehrere Roboter gebaut werden, die als Schwarm zusammenarbeiten können. Dafür mussten die Roboter modifiziert und mit einem zentralen Server verbunden werden.
Die Projektplanung bestand aus folgenden Schritten:
•	Aufgabenverteilung im Team
•	Erstellung eines Zeitplans mit allen wichtigen Meilensteinen
•	Planung des Materialbedarfs (Kits, Sensoren, Bauteile)
•	Organisation der Hardware- und Softwareentwicklung
Die Arbeitspakete wurden in einem Projektstrukturplan (PSP) festgehalten.
Jedes Teammitglied war für bestimmte Bereiche verantwortlich, z.B. Hardwareaufbau, Softwareprogrammierung oder Dokumentation.
%
\subsection{Controlling}
Um den Projektverlauf zu überwachen, wurde ein Controlling-System eingeführt. Dieses bestand aus drei Bereichen:
\subsubsection{Zeit-Controlling}
Es wurde ein Zeitplan mit Abgabeterminen und Zwischenschritten erstellt.
Die wichtigsten Termine waren:
•	Einkauf und Aufbau der Roboter
•	Fertigstellung der Hardware-Modifikationen
•	Abschluss der Softwareentwicklung
•	Durchführung der Tests
•	Abgabe der Dokumentation
Der Zeitplan wurde regelmäßig kontrolliert und angepasst, wenn es zu Verzögerungen kam.
\subsubsection{Kosten-Controlling}
Für das Projekt wurde eine Kostenaufstellung erstellt. Alle Ausgaben für Bauteile, Werkzeuge und Software wurden darin dokumentiert.
Die wichtigsten Kostenpunkte waren:
•	Elegoo Tumbller Kits
•	ESP32-Boards
•	LiDAR-Sensor
•	Zubehör (Kabel, Steckverbinder, etc.)
Am Ende wurde geprüft, ob das Projektbudget eingehalten wurde.
\subsubsection{Qualitäts-Controlling}
Die Qualität der Hardware und Software wurde während des gesamten Projekts durch Tests und Prüfungen sichergestellt.
Es wurde besonders darauf geachtet, dass:
•	alle Sensoren zuverlässig arbeiten
•	die Kommunikation zwischen Roboter und Server stabil ist
•	der Aufbau sauber und sicher ausgeführt wurde
%
\subsection{Problemlösung und Risikomanagement}
%
Während des Projekts traten verschiedene Probleme auf, die gelöst werden mussten:
Problem	Lösung
Bluetooth-Reichweite zu gering	Umstieg auf WLAN mit ESP32-Boards
Platzmangel für LiDAR-Sensor	Mechanische Anpassung des Guide-Roboters
Signalstörungen bei den Motoren	Überprüfung und Neuanordnung der Kabel
Instabile Verbindung bei hoher Serverlast	Optimierung der Websocket-Kommunikation
Fehlermeldungen in der Software	Implementierung von Debug-Tools und Log-Funktion
Verzögerungen bei der Hardwarelieferung	Anpassung des Zeitplans und Priorisierung der Arbeit
Alle Probleme wurden im Team besprochen und gelöst. Bei größeren Schwierigkeiten wurde ein Risiko-Logbuch geführt.
\subsection{Software-Management (SW)}
%
Die Softwareentwicklung war ein wichtiger Bestandteil des Projekts. Sie bestand aus folgenden Bereichen:
\subsubsection{Entwicklungsschritte}
1.	Server-Software:
Die Server-Software wurde mit Node.js und Websocket-Technologie umgesetzt. Sie sammelt die Sensordaten der Roboter und verteilt sie an alle Teilnehmer.
2.	Roboter-Software:
Die Roboter wurden mit einer Arduino-IDE programmiert. Die ESP32-Boards verwenden eine WLAN-Verbindung und kommunizieren über ein einfaches TCP/IP-Protokoll.
3.	Frontend (Benutzeroberfläche):
Zusätzlich wurde eine Webseite entwickelt, auf der die Positionen und Daten der Roboter in Echtzeit angezeigt werden.
\subsubsection{Versionskontrolle}
Für die Software wurde ein Git-Repository verwendet. Damit konnten Änderungen dokumentiert, rückgängig gemacht und im Team bearbeitet werden.
\subsubsection{Tests and Debugging}
Die Software wurde regelmäßig getestet:
•	Unit-Tests für einzelne Funktionen
•	Integrationstests für die Kommunikation
•	Live-Tests mit allen Robotern
\subsection{Dokumentation und Abschluss}
Zum Abschluss des Projekts wurden alle Ergebnisse in einer Diplomarbeit dokumentiert.
Dazu gehören:
•	Die Beschreibung der Hardware
•	Die Modifikationen
•	Die Software-Struktur
•	Der Projektverlauf
