% Projekthandbuch, Zeiterfassung? 
% Zuständig: Mihael
\section{Projektmanagment}
\initials{MS}
\label{subsec:projektmanagment}
%
\subsection{Projektplanung}
\initials{MS}
%

Zu Beginn des Projekts wurde das Ziel definiert, mehrere Roboter so umzubauen, dass sie gemeinsam als Schwarm arbeiten können. Die Roboter sollten untereinander kommunizieren, ihre Umgebung erkennen und zentral gesteuert werden. Um dieses Ziel erreichen zu können, wurde das Projekt in verschiedene Aufgabenbereiche aufgeteilt
,die unter den Teammitgliedern aufgeteilt wurden. Jeder war für bestimmte Bereiche verantwortlich, wie zum Beispiel den Aufbau der Hardware, die Programmierung oder die Dokumentation.
Zusätzlich wurde ein Projektstrukturplan (PSP) erstellt, um die einzelnen Arbeitsschritte übersichtlich darzustellen.
In der Planungsphase wurde außerdem festgelegt, welche Bauteile benötigt werden, wie die Zeitaufteilung erfolgen soll und wie die Zusammenarbeit im Team organisiert wird.
\subsection{Controlling}
\initials{MS}
Damit der Projektablauf nicht aus dem Ruder lief, wurde ein einfaches Controlling-System eingeführt. Dieses umfasste die Bereiche Zeit, Kosten und Qualität.
\subsubsection{Zeit-Controlling}
\initials{MS}
Ein Zeitplan wurde erstellt, in dem alle wichtigen Termine und Zwischenziele eingetragen waren.
Dazu gehörten unter anderem:
\begin{itemize}
    \item Bestellung und Aufbau der Roboter
    \item Abschluss der Modifikationen
    \item Fertigstellung der Software
    \item Durchführung der Testläufe
    \item Fertigstellung und Abgabe der Dokumentation
\end{itemize}
Dieser Plan wurde in einem GANTT-Chart dargestellt, das regelmäßig aktualisiert wurde.
Dadurch konnten die Auswirkungen von Verzögerungen in einzelnen Arbeitspaketen und die Folgen von Abhängigkeiten zwischen diesen verfolgt und abgeschätzt werden.
Siehe \ref{sec:Projektbalkenplan}.
\subsubsection{Kosten-Controlling}
\initials{MS}
Während der gesamten Projektzeit wurde eine Kostenaufstellung geführt. Alle Ausgaben für Bauteile, Werkzeuge und Zubehör wurden festgehalten.
Die größten Kostenpunkte waren die Elegoo Tumbller Kits, die neuen ESP32-Boards für die WLAN-Kommunikation, der LiDAR-Sensor für den Guide-Roboter und diverses Zubehör wie Kabel oder Schrauben.
Am Ende des Projekts wurde überprüft, ob das Budget eingehalten wurde.
\subsubsection{Qualitäts-Controlling}
\initials{MS}
Damit die Hardware und Software zuverlässig funktionieren, wurde während des gesamten Projekts auf die Qualität geachtet.
Es wurde regelmäßig kontrolliert, ob die Sensoren und Motoren korrekt arbeiteten, die Verbindung zwischen Server und Robotern stabil war und der Aufbau der Roboter sauber durchgeführt wurde.
Außerdem wurden regelmäßig Tests durchgeführt, um Fehler frühzeitig zu erkennen.
%
\subsection{Problemlösung und Risikomanagement}
\initials{MS}
%
Im Laufe des Projekts traten mehrere Probleme auf, die im Team besprochen und gelöst wurden.
Ein häufiges Problem war die geringe Reichweite der ursprünglichen Bluetooth-Kommunikation. Da diese für unser Projektziel nicht ausreichte, wurde beschlossen, auf WLAN umzusteigen und die Arduino Nano Boards durch ESP32-Boards zu ersetzen.
Diese Entscheidung führte zu einer Reihe von Anpassungsproblemen in Hard- und Software, 
die erhebliche Verzögerungen nach sich zogen.
Auch beim Umbau der Roboter kam es zu Schwierigkeiten. Für den LiDAR-Sensor des Guide-Roboters war im ursprünglichen Aufbau zu wenig Platz. Deshalb wurde das Chassis angepasst, indem die obere Ebene erhöht wurde.
Bei der Verkabelung der Motoren kam es anfangs zu Störungen. Die Kabel mussten neu verlegt werden, damit keine Signalprobleme auftreten.
Zusätzlich war die WLAN-Verbindung bei hoher Auslastung instabil. Dieses Problem wurde durch Anpassungen an der Server-Software und der Websocket-Kommunikation behoben.
Lieferverzögerungen bei einzelnen Bauteilen machten es notwendig, den Zeitplan zu ändern und die Arbeitsschritte neu zu priorisieren.
Alle auftretenden Probleme wurden in einer Risiko-Analyse festgehalten.
\subsection{Software-Management (SW)}
\initials{MS}
%
Ein wichtiger Bestandteil des Projekts war die Entwicklung der Software für die Roboter und den Server.
Die Softwareentwicklung gliederte sich in drei Bereiche:
\subsubsection{Entwicklungsschritte}
\initials{MS}
\begin{enumerate}
    \item \texttt{Server-Software:} \\
    Die Server-Software wurde mit dem JavaScript runtime environment Node.js geschrieben. Sie empfängt die Sensordaten aller Roboter und verteilt diese über Websocket-Verbindungen an alle Teilnehmer.
    Zusätzlich verwaltet der Server den aktuellen Zustand des Systems und ermöglicht es, alle Roboter zentral zu steuern.
    \item \texttt{Roboter-Software:} \\
    Die Roboter-Software wurde mit dem Microsoft Visual Studio Code und dem Plugin PlatformIO entwickelt.
    Die Steuerung der Roboter erfolgt über das ESP32-Board, welches eine WLAN-Verbindung zum Server herstellt.
    Die Roboter senden ihre Sensordaten an den Server und empfangen Steuerbefehle in Echtzeit.
    Die Kommunikation erfolgt über ein einfaches TCP/IP-Protokoll.
    \item \texttt{Frontend (Benutzeroberfläche):} \\
    Neben der Software für die Roboter und den Server wurde auch eine Benutzeroberfläche entwickelt.
    Auf dieser Webseite können die Positionen und Sensordaten der Roboter in Echtzeit angezeigt werden.
    Dadurch ist es möglich, den aktuellen Zustand des gesamten Schwarms schnell zu überblicken.  
    \end{enumerate}

\subsubsection{Versionskontrolle}
\initials{MS}
Damit die Softwareentwicklung nachvollziehbar bleibt, wurde ein Git-Repository eingerichtet.
Alle Änderungen an der Software wurden dort dokumentiert, sodass jederzeit auf ältere Versionen zurückgegriffen werden konnte.
Die Nutzung des Git-Repository stellte auch sicher, dass alle Projektmitglieder immer gegen den neuesten Software-Stand testeten.
\subsubsection{Tests and Debugging}
\initials{MS}
Die Software wurde während der gesamten Projektlaufzeit regelmäßig getestet.
Neben Funktionstests einzelner Programmbereiche wurde auch die Kommunikation zwischen den Robotern und dem Server getestet.
Zusätzlich wurden praktische Testläufe mit allen Robotern durchgeführt, um die Funktion des gesamten Systems zu überprüfen.
Zur besseren Fehlersuche wurden Debug-Ausgaben und Log-Funktionen in die Programme eingebaut.
\subsubsection{Dokumentation und Abschluss}
\initials{MS}
Am Ende des Projekts wurden alle Ergebnisse in dieser Diplomarbeit dokumentiert.
Die Diplomarbeit enthält die Beschreibung der Inhalte, die durchgeführten Modifikationen, die Software und den gesamten Ablauf des Projekts.
Ziel war es, alle Arbeitsschritte und Ergebnisse so festzuhalten, dass das Projekt jederzeit nachvollzogen werden kann.