% Allgemeine Probleme (Hard- und Software)

\section{Probleme}
\label{sec:probleme}
\subsection{Docker}
\label{subsec:probleme_docker}
%TODO wording?
Am Laptop von Herr Gastgeber war es am Beginn des Schuljahres nicht möglich,
jegliche Geräte im Netzwerk der HTL zu erreichen.
%
Grund dafür war eine Software namens Docker,
welche außerhalb dieses Projekts zur Containerisierung von anderen Anwendungen dient.
%
Docker war standardmäßig so konfiguriert,
dass es containerspezifische Subnets im IPv4-Adressbereich \allowbreak\texttt{172.117.0.0/16} erstellt.
%
Diese Subnets hatten dann am Laptop eine höhere Priorität als das Schulnetzwerk,
was dafür sorgte,
dass das weltweite Internet noch erreichbar war, 
nicht aber das lokale Schulnetzwerk.
%
Um Docker einen anderen IP-Adressbereich zuzuweisen,
wurde der Docker-Daemon mithilfe der Datei \texttt{/etc/docker/daemon.json}
wie in Listing \ref{lst:docker_address_pools} konfiguriert.
\begin{lstlisting}[language=json,gobble=4,
    label=lst:docker_address_pools,caption=Konfiguration für den Docker-Daemon]
    {
          "default-address-pools":
          [
            {"base":"10.10.0.0/16","size":24}
          ]
    }
\end{lstlisting}

\subsection{Modifikationen am Tumbller}
\label{subsec:problem_tumbbler_mods}
Als wir das Projekt geplant haben,
war die Idee,
einfach ein fertiges Kit ein bisschen zu modifizieren,
fast schon zu schön um wahr zu sein.
%
Und das war es dann auch.
%
Bei den Hardware-Modifikationen am Tumbller (siehe Kapitel \ref{subsec:elegoo_tumbller}) gab es zwei relativ große Probleme:

\subsubsection{Falsche Betriebsspannungen}
\label{subsec:problem_betriebsspannungen}
Beim Austausch des mitgelieferten Arduino Nano mit einem ESP32 im Nano-Format haben wir einen wichtigen Faktor übersehen:
%
Der Arduino Nano hat eine Betriebsspannung von 5V,
während der ESP32 mit 3.3V arbeitet.
%
Glücklicherweise funktionieren die meisten Komponenten des Elegoo Tumbllers auch mit 3.3V.
%
Die einzigen Bauteile,
welche eine höhere Spannung benötigen,
sind die auf der Platine angelöteten farbigen Leuchtdioden.
%
Um nicht das ganze Projekt von Grund auf neu aufbauen zu müssen,
haben wir uns entschieden,
dieses kleine optische Detail fürs Erste auszulassen.
%
%TODO schaltplan
In der originalen Beschaltung (siehe Abbildung \ref{fig:elegoo_tumbller_original_circuit})
wurde das Bluetooth-Modul mithilfe des Spannungswandlers \texttt{U3} mit 3.3V versorgt.
%
Diesen Spannungswandler haben wir ausgebaut und mittels einer Lötbrücke $V_{in}$ mit $V_{out}$ verbunden.
Allerdings erwies sich das Bluetooth Modul später als problematisch (siehe Abschnitt \ref{subsec:problem_bluetooth_serial}),
weshalb die Überbrückung eigentlich nicht notwendig ist.
\\\\
Außerdem war es beim Guide-Roboter aufgrund des Wechsels auf 3.3V nicht mehr möglich,
den LiDAR-Sensor direkt mit dem Spannungswandler des Mikrocontroller-Boards zu versorgen.
%
Deshalb haben wir für den Guide einen zusätzlichen Step-Down-Konverter eingebaut,
welcher die Versorgungsspannung des Akkus auf 5V für den LiDAR herunterregelt.
\subsection{Blockierte UART-Schnittstelle}
\label{subsec:problem_bluetooth_serial}
Als das Problem der inkompatiblen Spannungen gelöst,
und die erste Version des Programmes zum Testen der einzelnen Komponenten geschrieben war,
%TODO Formulierung: "großes Problem" hab ich schon vorher verwendet
sind wir auch schon auf das nächste nennenswerte Problem gestoßen:
%
Das externe Bluetooth-Modul,
was bereits auf der Platine verlötet war,
belegt die UART-Schnittstelle des eingebauten ESP32.
%
Das führte dazu, dass der ESP32 nur neu programmiert werden konnte,
wenn das ESP32-Devboard aus dem Roboter ausgebaut war.
%
Außerdem wurde dadurch das Debuggen mittels UART über USB unmöglich gemacht.
\\
Da wir die Roboter über WLAN steuern,
und der ESP32 auch ohne externe Erweiterungen bereits sowohl über WLAN- als auch Bluetooth-Kapazitäten verfügt,
haben wir entschlossen,
die mitgelieferte Platine weiter zu modifizieren,
indem wir das Bluetooth-Modul entfernen.
%
Außerdem haben wir den Transistor \texttt{Q1} und den Spannungsteiler,
welcher aus \texttt{R9} und \texttt{R10} besteht entfernt,
um sicherzustellen, das die UART-Verbindung keinesfalls beeinflusst wird
(siehe Original-Schaltplan in Abbildung \ref{fig:elegoo_tumbller_original_circuit}).
%TODO Bilder und Erläuterung zu Entfernen des Bluetooth Moduls