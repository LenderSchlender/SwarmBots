% Überblick zur technischen Struktur des Projektes
% z.B. Aufgaben der einzelnen Roboter, Kommunikationswege (protobufs!), Übersicht zu Algorithmen und Steuerung, etc.

\chapter{Technischer Überblick}
\label{sec:ueberblick}


\section{Unterstützende Programme}
\label{subsec:ueberblick_programs}

\subsection{Git}
\initials{LG}
Zur Versionskontrolle der Software und der Dokumentation einschließlich des vorliegenden Textes
(siehe Abschnitt \ref{subsec:latex}) wurde Git eingesetzt.
%
Das Repository befindet sich auf GitHub und ist unter \url{https://github.com/LenderSchlender/SwarmBots} öffentlich zugänglich (Stand \today).

\subsection{Syncthing}
\initials{LG}
Zum Teilen von anderen Dateien (Weekly Reports, Zeiterfassung, etc.) zwischen Personen und Geräten
wurde ein FOSS\footnote{Free and Open Source Software} Programm names Syncthing \cite{syncthing} genutzt.
%
Syncthing synchronisiert Dateien in einem Ordner dezentralisiert mittels Peer-to-Peer Verbindungen zwischen den Geräten.
%
Dadurch ist das Teilen von Dateien nicht von zentralisierter proprietärer Infrastruktur abhängig,
kann aber durch ``zentrale'' (und gut erreichbare) Server unterstützt werden.
%
Weiters stellt Syncthing eine rudimentäre Art der Versionskontrolle dar,
da es die Möglichkeit gibt,
bei Änderungen der Dateien eine alte Version der Datei für eine gewisse Zeit zu behalten.


Da Syncthing dezentralisiert ist,
und es somit keinen SPOT\footnote{Single Point of Truth} gibt,
kann es (selten, aber doch) zu Konflikten der Dateiversionen kommen (z.B. wenn zwei Personen die selbe Datei gleichzeitig bearbeiten). 
%
Bei solchen Konflikten erstellt Syncthing eine Kopie von einer der Versionen der Datei,
und der Nutzer muss selber entscheiden,
welche Version die ``richtige'' ist.

\subsection{Visual Studio Code}
\initials{LG}
Zur Bearbeitung der Quellcodes und der Diplomarbeit wird Visual Studio Code (auch VSC oder VSCode)
in Kombination mit mit einigen Erweiterungen verwendet.
%
Erwähnenswert ist hierbei auch die in VSC eingebaute Git-Integration zur Versionskontrolle,
welche keine zusätzliche Erweiterung benötigt.
\begin{table}[H]
    \centering
    \begin{tabular}{r|l}
        Erweiterung         & Verwendungszweck \\ \hline
        Python              & Programmierung des Backends \\
        PlatformIO IDE      & Programmierung der Roboter \\
        vscode-proto3       & Syntax-Highlighting für Protocol Buffers \\
        LaTeX Workshop      & LaTeX Unterstützung für VSC \\
        Code Spell Checker  & Rechtschreibprüfung \\
    \end{tabular}
    \caption{Verwendete VSCode-Erweiterungen}
    \label{tab:vsc_plugins}
\end{table}
\subsection{PlatformIO}
\initials{LG}
Für die Programmierung der Mikrocontroller wird PlatformIO \cite{platformio}
in Verbindung mit dem PlatformIO-Plugin für VSCode angewandt.
%
PlatformIO ist eine Alternative zur Arduino-IDE,
welche mithilfe von Plugins in viele IDEs wie z.B. VSCode oder CLion integriert werden kann.
%
Alternativ kann man PlatformIO auch über die Kommandozeile bedienen.
%
Vorteile von PlatformIO gegenüber der Arduino-IDE sind u.A. schnelleres Kompilieren,
ordentliche Auto-Vervollständigung,
statische Code-Analyse (Linting),
und ein schön geregeltes System für externe Bibliotheken.

\subsection{\LaTeX}
\label{subsec:latex}
\initials{LG}
Anstatt von WYSIWYG\footnote{What You See Is What You Get}-Editoren
wie Microsoft Word oder LibreOffice Writer wird zum Erstellen dieser Diplomarbeit
ein plattformunabhängiges Plaintext-Format namens LaTeX \cite{latex} verwendet.
% TODO not happy with wording above
Eine LaTeX-Datei ist (ähnlich wie Markdown) einfach nur eine Textdatei mit zusätzlichen Befehlen.
Deshalb ist es möglich, \texttt{.tex}-Dateien sehr gut mit Versionskontrollsoftware wie Git zu verwenden
und somit genau Änderungen rückzuverfolgen und gegebenenfalls zurückzusetzen.
%
Weiters ist es möglich, mit LaTeX das Dokument in mehrere Dateien aufzuteilen,
was das Ganze um einiges übersichtlicher macht.
%
Der größte Vorteil von LaTeX ist aber wohl,
dass die Formatierung einheitlich für den Autor übernommen wird,
sodass sich dieser besser auf den Inhalt konzentrieren kann.
%
In LaTeX wurden auch einige Makros definiert.
%
So wird der aktuelle Source-Code des Projekts automatisch direkt aus den
jeweiligen Dateien inkludiert (Siehe Abschnitt \ref{lstsec:incl-latex} auf Seite \pageref{lstsec:incl-latex}).
%
Außerdem wurde ein Makro definiert,
welches verwendet wird,
um am Anfang eines (Unter-)Kapitels die Initialien des zuständigen Hauptautors anzugeben. 

\section{Funktionsbeschreibung}
\label{subsec:funktionsbeschreibung}

\subsection{Guide}
\initials{LG}
Der Guide hat die Aufgabe,
mit dem nachgerüsteten LiDAR-Sensor die Umgebung nach Hindernissen abzusuchen.
%
Hierbei ist allerdings zu beachten,
dass der Guide die Datenverarbeitung nicht selbständig durchführt,
sondern die rohen Sensordaten einfach an das Backend weiterleitet.
%
In diesem Kontext sind Tamerlan und Bambi also nichts anderes als Hindernisse.

\subsection{Tamerlan \& Bambi}
\initials{LG}
Tamerlan und Bambi sind ``blind'' im dem Sinne,
dass sie über keine Sensoren zur Hinderniserkennung verfügen.
%
Sie sind komplett von Anweisungen des Backends abhängig.

\subsection{Backend}
\initials{LG}
Das Backend ist das ``Gehirn'' des Projekts.
%
Es empfangt die LiDAR-Daten vom Guide,
verarbeitet sie
und sendet Befehle an die Roboter.
%
Die LiDAR-Daten und die Ergebnisse der Verarbeitung werden auch
über eine Websocket-Verbindung ans Frontend weitergeleitet.
%
Außerdem empfängt es die Sensordaten der Gyroskope und Accelerometer der Roboter
und gibt diese auch an das Frontend weiter. 
%
Gegebenenfalls empfängt das Backend auch Befehle vom Frontend und leitet diese an die Roboter weiter.

\subsection{Frontend}
\initials{LG}
Das Frontend empfängt Sensordaten vom Backend und stellt diese dar.
%
LiDAR-Daten werden auf einer Karte dargestellt,
auf der auch die Ergebnisse der Hinderniserkennung am Backend zu sehen ist.
%
Die Daten der Gyroskope und Accelerometer werden in einem sich laufend aktualisierendem X/Y Diagramm dargestellt.
%
Zusätzlich gibt es im Frontend die Möglichkeit für Nutzer,
einen oder mehrere Roboter auf manuelle Steuerung umzuschalten
und diese(n) dann fernzusteuern.
%
Das Frontend ist eigentlich zum autonomen Betrieb nicht nötig,
es wird nur für Debugging-Zwecke und menschliche Intervention benötigt.

\subsection{Kommunikationswege}
\label{subsec:ueberblick_comms}
\initials{LG}
Die Kommunikation zwischen den einzelnen Systemkomponenten
funktioniert über das WebSocket-Protokoll (siehe \ref{sec:websockets}).
%
Tamerlan, Guide, und Bambi senden jeweils
Sensordaten an das Backend und empfangen Befehle von diesem.
%
Das Backend leitet die Sensordaten dann an das Frontend weiter,
wo sie angezeigt werden.
%
Bei manueller Steuerung werden die Steuerungsbefehle vom Frontend an das Backend gesendet,
welche diese an den jeweiligen Roboter weiterleitet. 
\begin{figure}[H]
    \centering
        \includegraphics[width=0.8\textwidth, center, trim={135 90 215 20}, clip]{img/Kommunikationswege.png}  
    \caption{Kommunikationswege}
    \label{fig:kommunikationswege}
\end{figure}

\section{WebSockets}
\label{sec:websockets}
\initials{LG}
WebSockets nach RFC6455 \cite{rfc6455} erlauben bidirektionale Kommunikation zwischen einem Client und einem Server.
%
Das WebSocket-Protokoll ist ein unabhängiges TCP-basiertes Verfahren;
seine einzige Beziehung zu HTTP besteht darin,
dass der WebSocket-Handshake als HTTP-Anfrage mit Antwort vom Typ
\texttt{101 Switching Protocols} nach RFC2616 \cite{rfc2616} interpretiert werden kann.
%
Standardmäßig verwendet das WebSocket-Protokoll Port 80 für normale WebSocket
Verbindungen und Port 443 für WebSocket-Verbindungen,
die mittels Transport Layer Security (TLS) verschlüsselt werden.
%
Dank des schon eben erwähnten HTTP-kompatiblen Handshakes
können auf dem selben Port sowohl HTTP- als auch WebSocket-Dienste zur Verfügung gestellt werden,
solange sich deren Pfad unterscheidet.
%
Natürlich muss hier die Serversoftware sowohl HTTP als auch WebSockets unterstützen.

Der große Vorteil von Websockets ist,
wie schon erwähnt,
der bidirektionale Informationsfluss:
%
Beide Verbindungsteilnehmer können ungefragt Informationen an den jeweils anderen Teilnehmer senden.
%
Bei HTTP müssen Informationen immer zuerst vom Client angefragt werden,
was insbesondere bei zeitkritischen Anwendungen wie Spielen,
Messenger-Apps,
oder eben bei der Fernsteuerung von Robotern von großem Nachteil ist.
%
Es ist zwar möglich,
mittels HTTP die Funktionalität von WebSockets anzunähern (long Polling),
jedoch gibt es dabei unter Anderem \cite{rfc6202} einen signifikant erhöhten Performance-Overhead%
\footnote{Bis zu einigen hundert Bytes,
  was im Kontext der SwarmBots ein Vielfaches des tatsächlichen Nachrichteninhalts ist.}
und erhöhte Komplexität.

In diesem Projekt werden WebSockets sowohl zur Kommunikation zwischen Robotern und Server,
als auch zwischen Frontend und Server verwendet.

\section{Protocol Buffers}
\label{subsec:ueberblick_protobufs}
\initials{LG}
Protocol Buffers \cite{protobufs} (``protobufs'') sind ein binäres Übertragungsformat,
welches von Google entwickelt und veröffentlicht wurde.
%
Gegenüber Datenformaten wie JSON und XML gibt es drei wesentliche Vorteile:
\begin{enumerate}
    \item Da Protocol Buffers auf Binärdaten anstatt von Text basieren,
    ist die Übertragung viel effizienter \cite{7765670}.
    %
    Insbesondere bei der Verwendung mit Mikrocontrollern ist das ein enormer Vorteil.

    \item Bei Protocol Buffers gibt es explizit definierte Datenstrukturen.
    %
    Diese Datenstrukturen sind (bei korrekter Verwendung) mit älteren Versionen rückwärts-kompatibel.

    \item Für die Verwendung mit unterschiedlichen Programmiersprachen kann (und soll) man aus protobuf-Definitionen
    Wrapper-Bibliotheken generieren.
    %
    Diese Wrapper-Bibliotheken können ohne weiteren Aufwand direkt verwendet werden,
    um auf die Datenstrukturen zuzugreifen.
\end{enumerate}

\subsubsection{Effizienz}
\initials{LG}
Wie schon oben erwähnt erreichen Protocol Buffers,
insbesondere bei kleinen Nachrichten,
einen viel kleineren Overhead als textbasierte Formate wie JSON oder XML.
%
Als Beispiel hierfür soll eine Nachricht vom Typ \texttt{EncoderData} gelten:
%
Mittels Protocol Buffers wird eine solche Beispielnachricht wie folgt übertragen:
%
\begin{lstlisting}[label=lst:testnachricht-proto,caption=Testnachricht (Protobuf) in Hexadezimaldarstellung]
    0890013205082a109c12
\end{lstlisting}
So ist die Testnachricht im Protobuf-Format lediglich 10 Bytes lang!
%
Zum Vergleich ist die gleiche Nachricht im JSON-Format um einiges länger:
\begin{lstlisting}[label=lst:testnachricht-json,caption=Testnachricht (JSON) als formattierter String]
{
  "seq": 144,
  "msg": {
    "pulses": 42,
    "duration": 2332
  }
}
\end{lstlisting}
So wie in Listing \ref{lst:testnachricht-json} dargestellt,
werden 71 Bytes benötigt.
% TODO im?
Da im JSON-Format Leerzeichen ignoriert werden können,
ist es möglich, die Größe der Nachricht auf 51 Bytes zu reduzieren.
%
Das ist immer noch mehr als fünf mal so viel wie mit Protocol Buffers.

Der Unterschied in Nachrichtengröße ist stark von der Nachricht selbst abhängig;
Bei größeren Nachrichten schrumpft der Overhead von JSON
und damit der ``Vorsprung'' von Protocol Buffers --
allerdings sind Protocol Buffers dennoch effizienter als JSON.
%
Zu beachten ist auch,
dass bei JSON die Länge der Variablennamen einen Einfluss auf die Nachrichtengröße hat.
%
Für dieses Beispiel wurden die Bezeichnungen \texttt{sequence} und \texttt{message} absichtlich auf
\texttt{seq} respektive \texttt{msg} abgekürzt,
um die Testbedingungen etwas fairer zu gestalten.
%
Bei Verwendung von WebSockets ist JSON mehr als adäquat,
jedoch führen die Ineffizienzen des Formats bei anderen Protokollen,
wie ESP-NOW \cite{esp-now} (welches nur Nachrichten mit bis zu 250 Bytes unterstützt),
zu Problemen.
%
Um also die Möglichkeit zur Erweiterung auf andere Protokolle zu gewährleisten,
wurde auf die Vorteile von JSON verzichtet.

\subsubsection{Datenstruktur}
\initials{LG}
Die jeweiligen Nachrichten (``Payloads'')
werden vom Nachrichtensender gemeinsam mit einer Sequenznummer in einem \texttt{Wrapper} verpackt,
welcher dann über die WebSocket-Verbindung an den Nachrichtenempfänger übertragen wird.
%
Der Empfänger speichert immer die letzte empfangene Sequenznummer
und verwirft Nachrichten,
welche eine niedrigere Sequenznummer als die gespeicherte haben.
%
Eigentlich ist dies aufgrund der Verwendung von WebSockets nicht nötig,
da TCP-Pakete aufgrund einer eigenen Sequenznummer immer garantiert in der richtigen Reihenfolge ankommen.
%
Um die Verwendung des Protokolls mit anderen unterliegenden Transporten,
wie zum Beispiel UDP oder ESP-NOW,
zu erlauben,
wurde zusätzlich die protokolleigene Sequenznummer implementiert.
%
Genaue Informationen zum Datenformat können in Abschnitt
\ref{lstsec:incl-protobuf} auf Seite
\pageref{lstsec:incl-protobuf} gefunden werden.

\subsubsection{Wrapper-Bibliotheken}
\initials{LG}
Zur Verwendung von Protocol Buffers in Programmen gibt es eine Vielzahl von Wrapper-Bibliotheken,
welche es ermöglichen,
mit dem in den \texttt{.proto}-Dateien definiertem Format
Nachrichten zu erstellen und zu verarbeiten.
%
Da in diesem Projekt mehrere Programmiersprachen verwendet wurden,
wurden für die einzelnen Programme natürlich auch unterschiedliche Bibliotheken genutzt:
%
Für die Implementierung der Roboter in C++ wurde die Bibliothek \texttt{Nanopb} verwendet,
welche insbesondere für die Verwendung in Systemen mit limitierten Ressourcen gedacht ist.
%
Im Server wurde die Python-Bibliothek \texttt{protobuf} verwendet,
während am Frontend \texttt{protobufjs} mit eingebautem TypeScript-Support angewandt wurde.

Zur automatischen Generierung der Dateien zur Verwendung von Protobufs in der jeweiligen Programmiersprache
wurden von Herrn Gastgeber zwei kleine Skripts für die Roboter und das Frontend entwickelt,
welche bei Bedarf die jeweiligen Bibliotheksfunktionen aufrufen.
%
Beim Server ist leider das Ausführen eines Befehls und die manuelle Modifikation einer Datei notwendig,
wie in der Datei \texttt{server/README.md} dokumentiert. (Siehe Abschnitt \ref{lstsec:incl-server} auf Seite \pageref{lstsec:incl-server} ff.) 

\section{Sensoren}
\label{subsec:ueberblick_sensors}
Mehr Details (und Bilder) zu den Sensoren und den anderen Hardwarekomponenten sind im
Abschnitt \ref{sec:hardware} auf Seite \pageref{sec:hardware} ff. zu finden.

\subsection{LiDAR}
\label{subsec:ueberblick_lidar}
\initials{LG}
Der LiDAR-Sensor ermöglicht es dem \textit{Guide},
einen ``Rundumblick'' auf die Umgebung zu werfen.
%
Dies allerdings nur auf einer Ebene,
da der verwendete Sensor (\texttt{youyeetoo LD20})
nur ein 2D-LiDAR ist.
%
Trotz dieser Einschränkungen ist ein 360° LiDAR-Sensor aufgrund der Möglichkeit,
Entfernungen in jede mögliche Richtung zu messen,
wesentlich besser als ein Ultraschallsensor.
%
Außerdem ist die Genauigkeit von LiDAR-Sensoren im Allgemeinen
um ein Vielfaches höher als bei ultraschallbasierten Alternativen.
%
Um den 360°-Rundumblick zu gewähren,
muss der LiDAR-Sensor am höchsten Punkt des Roboters installiert werden,
da sonst Teile des Blickfelds
durch andere Komponenten des Roboters verdeckt werden könnten.

\subsection{Gyroskop}
\initials{LG}
Jeder der drei Roboter ist mit einem Gyroskop/Accelerometer vom Typ \texttt{MPU6050} ausgestattet.
%
Dieser Sensor ist bereits im originalen Tumbller-Kit inkludiert, also ist die Installation entsprechend einfach.
%
Der \texttt{MPU6050} versorgt den Mikrocontroller über I2C mit Messdaten zu drei Gyroskop- und drei Accelerometerachsen.
%
Diese Informationen werden dann vom Mikrocontroller verwendet, um den Roboter aufrecht zu halten.
%
Weiters leitet der Mikrocontroller die Daten zur weiteren Verarbeitung bzw. Analyse für Debugging-Zwecke an den zentralen Server weiter.

\subsection{Drehgeber}
\label{subsec:ueberblick_rot_enc}
\initials{LG}
In den im Tumbller-Kit mitgelieferten Motoren ist bereits ein Drehgeber mit eingebaut,
welcher mit Hall-Sensoren funktioniert.
%
Mit diesem Drehgeber kann die aktuelle Motorgeschwindigkeit als Feedback zur Motorsteuerung gemessen werden.
%
Die Impulse des Drehgebers werden für eine gewisse Zeitspanne zusammengefasst und dann regelmäßig an den Server weitergeleitet.