% Überblick zur technischen Struktur des Projektes
% z.B. Aufgaben der einzelnen Roboter, Kommunikationswege (protobufs!), Übersicht zu Algorithmen und Steuerung, etc.
% TODO Erklärung VSC (plugins!), Git, Syncthing

\section{Technischer Überblick}
\label{sec:ueberblick}


\subsection{Unterstützende Programme}
\label{subsec:ueberblick_programs}

\subsubsection{Git}
Zur Versionskontrolle der Software und der Diplomarbeit selber (siehe Abschnitt \ref{subsec:latex}) haben wir Git eingesetzt.
%
Das Repository ist auf GitHub, aber aktuell privat.

\subsubsection{Syncthing}
Zum Teilen von anderen Dateien (Weekly Reports, Zeiterfassung, etc.) zwischen Personen und Geräten
haben wir ein FOSS\footnote{Free and Open Source Software} Programm names Syncthing \cite{syncthing} genutzt.
%
Syncthing synchronisiert Dateien in einem Ordner dezentralisiert mittels Peer-to-Peer Verbindungen zwischen den Geräten.
%
Dadurch ist das Teilen von Dateien nicht von zentralisierter proprietärer Infrastruktur abhängig,
kann aber durch ``zentrale'' (und gut erreichbare) Server unterstützt werden.
%
Weiters stellt Syncthing eine rudimentäre Art der Versionskontrolle dar,
da es die Möglichkeit gibt,
bei Änderungen der Dateien eine alte Version der Datei für eine gewisse Zeit zu behalten.


Da Syncthing dezentralisiert ist,
und es somit keinen SPOT\footnote{Single Point of Truth} gibt,
kann es (selten, aber doch) zu Konflikten der Dateiversionen kommen (z.B. wenn zwei Personen die selbe Datei gleichzeitig bearbeiten). 
%
Bei solchen Konflikten erstellt Syncthing eine Kopie von einer der Versionen der Datei,
und der Nutzer muss selber entscheiden,
welche Version die ``richtige'' ist.

\subsubsection{Visual Studio Code}
Zur Bearbeitung der Quellcodes und der Diplomarbeit verwenden wir Visual Studio Code (auch VSC oder VSCode) mit einigen Erweiterungen.
%
Nützlich ist für uns auch die Git-Integration zur Versionskontrolle.
\begin{table}[H]
    \centering
    \begin{tabular}{r|l}
        Erweiterung         & Verwendungszweck \\ \hline
        Python              & Programmierung des Backends \\
        PlatformIO IDE      & Programmierung der Roboter \\
        vscode-proto3       & Syntax-Highlighting für Protocol Buffers \\
        LaTeX Workshop      & LaTeX Unterstützung für VSC \\
        Code Spell Checker  & Rechtschreibprüfung \\
    \end{tabular}
    \caption{Verwendete VSCode-Erweiterungen}
    \label{tab:vsc_plugins}
\end{table}
\subsubsection{PlatformIO}
Für die Programmierung der Mikrocontroller verwenden wir PlatformIO \cite{platformio} in Verbindung mit dem PlatformIO-Plugin für VSCode.
%
PlatformIO ist eine Alternative zur Arduino-IDE,
welche mithilfe von Plugins in viele IDEs wie z.B. VSCode oder CLion integriert werden kann.
%
Alternativ kann man PlatformIO auch über die Kommandozeile bedienen.
%
Vorteile von PlatformIO gegenüber der Arduino-IDE sind u.A. schnelleres Kompilieren,
ordentliche Auto-Vervollständigung,
statische Code-Analyse (Linting),
und ein schön geregeltes System für externe Bibliotheken.

\subsubsection{\LaTeX}
\label{subsec:latex}
Anstatt von WYSIWYG\footnote{What You See Is What You Get}-Editoren
wie Microsoft Word oder LibreOffice Writer verwenden wir zum Erstellen dieser Diplomarbeit
ein plattformunabhängiges Plaintext-Format namens \LaTeX.
% TODO not happy with wording above
Eine LaTeX-Datei ist (ähnlich wie Markdown) einfach nur eine Textdatei mit zusätzlichen Befehlen.
Deshalb können wir \texttt{.tex}-Dateien sehr gut mit Versionskontrollsoftware wie Git verwenden
und somit genau Änderungen rückverfolgen und gegebenenfalls zurücksetzen.
%
Weiters können wir mit LaTeX unsere Arbeit in mehrere Dateien aufteilen,
was das Ganze um einiges übersichtlicher macht.
%
Der größte Vorteil von LaTeX ist aber wohl,
dass die Formatierung einheitlich für den Nutzer übernommen wird,
sodass sich dieser besser auf den Inhalt konzentrieren kann.
%
Wenn man aber dann doch mal selber etwas anders formatieren will,
ist das mit LaTeX durch die Verwendung von Textbefehlen
aber immer noch schneller als in MS Word zur Maus zu greifen
und sich durch GUIs durchzuklicken. 

\subsection{Kommunikationswege}
\label{subsec:ueberblick_comms}

\subsubsection{Protocol Buffers}
\label{subsec:ueberblick_protobufs}
Protocol Buffers \cite{protobufs} (``protobufs'') sind ein binäres Übertragungsformat,
welches von von Google entwickelt und veröffentlicht wurde.
%
Gegenüber Datenformaten wie JSON und XML gibt es drei wesentliche Vorteile:
\begin{enumerate}
    \item Da Protocol Buffers auf Binärdaten anstatt von Text basieren,
    ist die Übertragung viel effizienter \cite{7765670}.
    %
    Insbesondere bei der Verwendung mit Mikrocontrollern ist das ein enormer Vorteil.

    \item Bei Protocol Buffers gibt es explizit definierte Datenstrukturen.
    %
    Diese Datenstrukturen sind (bei korrekter Verwendung) mit älteren Versionen rückwärts-kompatibel.

    \item Für wie Verwendung mit unterschiedlichen Programmiersprachen kann (und soll) man aus protobuf-Definitionen
    Wrapper-Bibliotheken generieren.
    %
    Diese Wrapper-Bibliotheken können ohne weiteren Aufwand direkt verwendet werden,
    um auf die Datenstrukturen zuzugreifen.
\end{enumerate}

\paragraph{Effizienz}
Wie schon oben erwähnt erreichen Protocol Buffers,
insbesondere bei kleinen Nachrichten,
einen viel kleineren Overhead als textbasierte Formate wie JSON oder XML.
%
% Quelle: https://protobuf.dev/programming-guides/encoding/
TODO mehr infos; Tatsächlicher Vergleich mit Zahlen.


\paragraph{Datenstrukturen}
TODO

\paragraph{Wrapper-Bibliotheken}
TODO

\subsection{Sensoren}
\label{subsec:ueberblick_sensors}

\subsubsection{LiDAR}
\label{subsec:ueberblick_lidar}

\subsubsection{Gyroskop}
\label{subsec:ueberblick_gyro}

\subsubsection{Dreh-Encoder}
\label{subsec:ueberblick_rot_enc}