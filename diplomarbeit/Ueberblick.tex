% Überblick zur technischen Struktur des Projektes
% z.B. Aufgaben der einzelnen Roboter, Kommunikationswege (protobufs!), Übersicht zu Algorithmen und Steuerung, etc.
% TODO Erklärung VSC (plugins!), Git, Syncthing

\section{Technischer Überblick}
\label{sec:ueberblick}


\subsection{Unterstützende Programme}
\label{subsec:ueberblick_programs}

\subsection{Kommunikationswege}
\label{subsec:ueberblick_comms}

\subsubsection{Protocol Buffers}
\label{subsec:ueberblick_protobufs}
Protocol Buffers \cite{protobufs} (``protobufs'') sind ein binäres Übertragungsformat,
welches von von Google entwickelt und veröffentlicht wurde.
%
Gegenüber Datenformaten wie JSON und XML gibt es drei wesentliche Vorteile:
\begin{enumerate}
    \item Da Protocol Buffers auf Binärdaten anstatt von Text basieren,
    ist die Übertragung viel effizienter \cite{7765670}.
    %
    Insbesondere bei der Verwendung mit Mikrocontrollern ist das ein enormer Vorteil.

    \item Bei Protocol Buffers gibt es explizit definierte Datenstrukturen.
    %
    Diese Datenstrukturen sind (bei korrekter Verwendung) mit älteren Versionen rückwärts-kompatibel.

    \item Für wie Verwendung mit unterschiedlichen Programmiersprachen kann (und soll) man aus protobuf-Definitionen
    Wrapper-Bibliotheken generieren.
    %
    Diese Wrapper-Bibliotheken können ohne weiteren Aufwand direkt verwendet werden,
    um auf die Datenstrukturen zuzugreifen.
\end{enumerate}

\paragraph{Effizienz}
Wie schon oben erwähnt erreichen Protocol Buffers,
insbesondere bei kleinen Nachrichten,
einen viel kleineren Overhead als textbasierte Formate wie JSON oder XML.
%
% Quelle: https://protobuf.dev/programming-guides/encoding/
TODO mehr infos; Tatsächlicher Vergleich mit Zahlen.


\paragraph{Datenstrukturen}
TODO

\paragraph{Wrapper-Bibliotheken}
TODO

\subsection{Sensoren}
\label{subsec:ueberblick_sensors}

\subsubsection{LiDAR}
\label{subsec:ueberblick_lidar}

\subsubsection{Gyroskop}
\label{subsec:ueberblick_gyro}

\subsubsection{Dreh-Encoder}
\label{subsec:ueberblick_rot_enc}