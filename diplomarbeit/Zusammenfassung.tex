% Zusammenfassung/Abstract

\chapter{Zusammenfassung}
Ziel dieser Diplomarbeit ist es, drei Roboter zu entwickeln,
welche kooperativ die Umgebung erkunden können.
%
Hierbei ist ein Roboter (``Guide'') mit einem LiDAR-Sensor ausgestattet,
welcher rundum Entfernungsmessungen durchführt.
%
Die anderen beiden Roboter (getauft ``Tamerlan'' und ``Bambi'')
sollen komplett ``blind'' sein.
%
Die Entfernungsmessungen des LiDAR-Sensors werden
gemeinsam mit allen anderen gesammelten Sensordaten
(Gyroskop, Accelerometer, Thermometer, Drehgeber)
über WebSocket-Verbindungen an einen zentralen Server gesendet,
welcher die Daten aller Roboter in Echtzeit verarbeitet.
%
Ziel ist es, dass der Server aufgrund dieser Daten
vollautomatische Entscheidungen treffen kann und die Roboter dementsprechend fernsteuert.
%
Alternativ können die Roboter auch mittels einem Webinterface manuell ferngesteuert werden.
%
Im Webinterface werden zusätzlich zu den gesammelten Sensordaten
auch Livestreams von Kameras dargestellt,
welche auf den Robotern montiert werden.
%
Diese Livestreams dienen nur der menschlichen Interpretation,
sie werden nicht automatisch ausgewertet.
%
Als zusätzliche Aufgabe sollen sich die Roboter auf nur einer Radachse balancieren,
da Kits für balancierende Roboter verwendet werden,
welche teilweise modifiziert werden,
um den Anforderungen des Projekts gerecht zu werden.
%
Die Software basiert auf den Robotern,
die letztes Jahr im Zuge der Projektwoche als Vorbereitung auf die Diplomarbeit gebaut wurden.

\section{Abstract (English)}
Goal of this diploma thesis is to develop three robots which
can explore the environment cooperatively.
%
One robot (``Guide'') is equipped with a LiDAR sensor,
which carries out distance measurements.
%
The other two robots (named ``Tamerlan'' and ``Bambi'')
are to be completely ``blind''.
%
The distance measurements from the LiDAR sensor and all other collected sensor data
(gyroscope, accelerometer, thermometer, rotary encoder)
is sent to a central server via WebSocket connections.
This server processes the data from all robots in real time.
%
The aim is for the server to be able to make fully automated decisions based on this data,
and control the robots remotely accordingly.
%
Alternatively, the robots can also be remote-controlled manually via a web interface.
%
In addition to the collected sensor data, the web interface
also displays live streams from cameras,
which are mounted on the robots.
%
These live streams are for human interpretation only,
they are not evaluated automatically.
%
As an additional task, the robots should balance themselves on only one wheel axis,
as kits for balancing robots were used,
which have been modified for our purposes.
The software is based on the robots
that were built last year during the project week in preparation for the diploma thesis.