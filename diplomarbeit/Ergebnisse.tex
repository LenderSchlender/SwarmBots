\chapter{Ergebnisse}

\section{Hardware}
Die Hardware der Roboter wurde fertiggestellt.
Der ursprüngliche Aufbau des Elegoo Tumbler wurde modifiziert, siehe \ref{subsec:hardware_modifikationen}
Da durch den Austausch vom Arduino Nano zum ESP32 Nano und dem Einbau des LiDAR am Guide
mehrere Modifikationen nötig waren,
wie es auch im Abschnitt \ref{subsec:hardware_modifikationen} und \ref{sec:probleme} nachlesbar ist.
\section{Software}

\subsection{Roboter}
\initials{LG}
Die Software der Roboter wurde größtenteils fertiggestellt.
%
Die alte Core-Bibliothek von der Projektwoche 2023/24 (Abschnitt \ref{subec:robots_core})
wurde erfolgreich auf die Verwendung mit dem Elegoo Tumbller modifiziert.
%
Die Kommunikation und Steuerung mittels Protocol Buffers,
welche über WebSockets versendet werden,
wurde auf Seiten der Roboter ebenfalls vollständig implementiert.
% TODO ist balancieren fertig geworden?
%
Die Steuerung der standardmäßig im Tumbller eingebauten LEDs wurde nicht fertiggestellt.
\subsection{Server}
\initials{JS}
Die Software des Backends ist größtenteils in der Entwicklung 
oder noch nicht implementiert.
% 
Die Kommunikation exklusiv zwischen Frontend und Roboter 
über das Backend ist funktionsfähig.
% 
Erkennungssystem, Datenbank und Datenverarbeitung sind nicht funktionsfähig.
% 
% TODO ist manuelle Steuerung funktionsfähig?
% 
Mehr zum Backend stand siehe Abschnitt \ref{subsec:backend_aktueller_stand} 
auf Seite \pageref{subsec:backend_aktueller_stand}.
% Verschiebe teile zu Ergebnisse?
% 
% Wie ist das bereitstellen des Frontends über den Server geregelt? - JS
\subsection{Frontend}
\initials{AB}
Der Code des Frontends ist nahezu vollständig implementiert.
%
Einige Diagramme für die anderen Sensor-Daten müssen noch erstellt werden,
sowie die LiDAR-Karte oder die Darstellung der Kamera-Bilder.

Das Frontend kann mit dem Server über den WebSocket kommunizieren,
und empfangenen Daten auf Diagrammen darstellen. 
%
Das Frontend hat eine Startseite, welche als Einführung in die Diplomarbeit dient.

\subsection{Kameras}
\initials{LG}
Die (relativ simple) Software der ESP32-CAM Module (Abschnitt \ref{subsec:robots_cams})
wurde wie geplant ausgearbeitet.
%
Der Code dazu kann in Abschnitt \ref{lstsec:incl-cam} auf Seite \pageref{lstsec:incl-cam} gefunden werden.