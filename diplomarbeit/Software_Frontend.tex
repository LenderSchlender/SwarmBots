% Software Frontend (HTML/CSS/JS; Vue.js)
% Zuständig: Arthur

\section{Software - Frontend}
\label{sec:software_frontend}
Das Frontend wurde als Webanwendung realisiert,
welche die vom Server gesammelten Daten visualisiert.
%
Dazu zählen z.B. die Daten vom LiDAR-Sensor,
vom Beschleunigungssensor
und die Geschwindigkeit der Roboter,
welche mithilfe der Encoder ausgelesen wird.
%
Um die Aktivitäten der Roboter mitverfolgen zu können,
ohne vor Ort anwesend sein zu müssen,
werden auch die Livestreams von Kameras,
welche auf den Robotern angebracht sind,
angezeigt.

Das Frontend wird mithilfe von Vue.js programmiert.
%
Vue.js ist ein JavaScript-Framework zur Webentwicklung
und bietet verschiedene Möglichkeiten,
Webseiten zu programmieren.
%
Bei Vue.js kann man zwischen der ``Options'' oder
der neueren ``Composition'' API unterscheiden.
%
Die Composition API ist etwas flexibler in der Anwendung,
während die Options API eine klar definierte Struktur verfolgt.
%
Letztlich ist es dem Entwickler überlassen,
je nach seinen Präferenzen zu wählen.
%
Wir verwenden für unsere Anwendung die Composition API,
da diese etwas intuitiver ist
und ``normalem'' JavaScript etwas näher kommt. 
%
Der große Vorteil des Vue.js Frameworks ist,
dass man einzelne Komponenten einer Website
als sogenannte SFC\footnote{Single-File Components}s definiert.
%
Wie der Name schon impliziert ist in diesem SFC alles enthalten,
was diese Komponente braucht: HTML, CSS und Type-/JavaScript.
%
Dadurch wird der Code schön strukturiert und klar aufgeteilt.
%
Des Weiteren können Vue-Komponenten auch
in anderen Komponenten wiederverwendet werden,
was eine schöne Abstrahierung ermöglicht.

\subsection{LiDAR-Karte}
\label{subsec:frontend_lidar_map}
Die LiDAR-Karte wird mithilfe von Vue.js auf dem Frontend, der Webseite generiert und soll Hindernisse wie Wände, Säulen oder auch Menschen mit farbigen Punkten anzeigen. 
Neben der verschiedenen Hindernisse, die auf der Karte einzusehen sind, sollen auch die Position der Roboter markiert werden. 
\begin{figure}[H]
    \includegraphics[width=0.4\textwidth, center]{img/LiDARMessungZeichnung_alt.png}
    \caption{LiDAR-Messung}
    \label{fig:LiDAR-Messung}
\end{figure}
Das Bild zeigt die generierte Punktwolke des LiDAR-Sensors. Die Punktwolke zeigt ideal die Funktionsweise des LiDARs und ist leicht zu interpretieren. 
Die unterschiedlichen Laserimpulse, welche gesendet werden, reflektieren an der Oberfläche und werden gemessen. Nicht nur ist die Entfernung, sondern auch die Tiefenwahrnehmung somit feststellbar.

Um die Punktwolke genauer zu verstehen, wurden Sektionen eingezeichnet. 
Die Grüne Sektion zeigt Objekte, welche unmittelbar vor dem LiDAR standen, in dem Fall waren es Büromaterialien wie etwa Kugelschreiber, Hefte, etc.
Die blaue Sektion befindet sich ca. 1-2 m vom LiDAR entfernt. Die aufgenommenen Punkte sind einfach nur eine Person sowie ein Sessel.
Die gelbe Sektion war ca. 3 m entfernt und zeigt die Umrandung des Raumes, interessanterweise sind die Lücken die Fenster, wo die Laserimpulse nicht reflektiert worden sind
und somit konnten keine Punkte an der Stelle der Wolke generiert werden konnten.

\subsection{Fernüberwachung per Kamera}
\label{subsec:frontend_cam_stream}
Auf allen drei Robotern befinden sich fest montierte ESP32-Kameras zur Überwachung, welche auf dem Frontend, der Webseite gestreamet werden.
Die Kameras dienen vor allem nur der Überwachung der Roboter und sollen der Gruppe nur helfen, Gefahren rechtzeitig zu identifizieren, um die Sicherheit des Projektes zu gewährleisten. Eingriffe erfolgen nur sofern die Roboter nicht selbst auf die Gefahr reagieren.
Sollte jedoch ein Fehler bei einem Roboter unterlaufen sein, so können wir mithilfe der Fernsteuerung (Siehe Abschnitt \ref{subsec:frontend_control}) probieren, die Roboter aus der Gefahrenzone zu entfernen.

\subsection{Fernsteuerung}
\label{subsec:frontend_control}
Die Idee der Fernsteuerung wurde aus den Prototypen der Projektwoche aus dem Jahr 2023/2024 übernommen. Diese hatten eine Fernsteuerungsfunktion, die hier in der Diplomarbeit nur in Notfällen oder Testzwecken verwendet wird.  

\subsection{Anzeigen der Sensordaten}
\label{subsec:frontend_sensors}
Die Sensordaten werden alle auf dem Frontend dargestellt und stetig aktualisiert. Die Sensoren, welche die Gruppe verwendet, sind der LiDAR, die Beschleunigungssensoren und die Kompasse.
Der LiDAR erstellt eine Karte mithilfe von einer Punktwolke, um die Hindernisse und Entfernungen feststellen zu können. Für mehr Informationen siehe Abschnitt \ref{subsec:frontend_lidar_map}.
%TODO BILD ÄNDERN
\begin{figure}[H]
    \includegraphics[width=0.4\textwidth, center]{img/Sensor_Anzeige_LOESCHEN.png}
    \caption{Anzeige der Sensordaten}
    \label{fig:Sensordaten}
\end{figure}
Die Beschleunigungssensoren werden in Graphen abhängig von der Zeit angezeigt, um die aktuellen Werte mit den vergangenen Werten vergleichen zu können. Nicht nur werden die Werte in Graphen abgebildet, sondern auch "normal" als Dezimalzahl
um schnell den aktuellsten Wert lesen zu können.
\\
Die Kompasse zeigen die aktuelle Fahrtrichtung der Roboter an. Geplant ist ein "realer" 2D-Kompass, wo die jedoch die Nadel die Fahrtrichtung anzeigt, statt wie gewohnt den Norden.